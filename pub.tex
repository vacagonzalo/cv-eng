\section{Proyectos Seleccionados}

\newcommand{\pub}[5]{
	\parbox[t][][t]{\linewidth}{%
		\begin{small}
		\parbox{\linewidth}{{``#4''}}
		\smallbreak
		\parbox{\linewidth}{{#2}, {#1}}
		%\parbox{\linewidth}{\small{Keywords:}\footnotesize{ #5}}
		\parbox{\linewidth}{{\href{https://doi.org/#3}{#3}}}
		\end{small}
	}
	\bigbreak
	\smallskip
}

\pub{Oct. 2022}{Linux driver for AXI4 slave}{}{A kernel space driver that manages transactions with an AXI slave synthesized in programmable logic.}{
}

\pub{Aug. 2022}{Real Time Operative System \emph{myOS}}{}{A real-time operating system for the Cortex-M4 architecture made from scratch. It implements a round robin scheduler with priorities, interrupt handling and a static stack system for each task.}{
}

\pub{Jun. 2022}{Microcontroller evaluator for space missions}{}{A set of tools that allows to inject SEFI-SEU and obtain the figure of merit of a microcontroller. Project promoted by INVAP within the framework of the Master's Degree in Internet of Things of the University of Buenos Aires.}{
}

\pub{May. 2021}{Environmental monitoring integrated to Honeywell's EBI}{}{A flexible system that joins the MQTT protocol for a network of sensors and presents them in MODBUS protocol to the proprietary environment that is operating in the plant, carried out within the framework of the IoT Specialization of the University of Buenos Aires}{
}

%\pub{Aug. 2018}{SRMPDS (Best paper)}{10.1145/3229710.3229759}{Flexible device sharing in PCIe clusters using
%Device~Lending}{%
%Paravirtualizaiton; KVM, Virtual Machines;
%}

